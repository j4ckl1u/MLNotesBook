% !Mode:: "TeX:UTF-8"

\section{GMM}

高斯混合模型的定义为若干高斯模型的线性加权,如下:
\begin{displaymath}
\begin{split}
&p(z_k)=\pi_k\\
&p(x_{i}|z_k)=\mathcal{N}(x_{i}|\mathbf{\mu}_k,\mathbf{\varSigma}_k) = 
\frac{exp\Big{\{}
-\frac{1}{2}(x_i-\mu_k)^T\varSigma_k^{-1}(x_i-\mu_k)
\Big{\}}}{(2\pi)^{\frac{D}{2}}|\varSigma_k|^{\frac{1}{2}}} 
\\
&p(x_{i})=\sum_{k}^{K}{p(z_k)}p(x_{i}|z_k)=\sum_{k=1}^{K}\pi_k\mathcal{N}(x_{i}|\mathbf{\mu}_k,\mathbf{\varSigma}_K)\\
&p(z_k|x_{i})
=\frac{p(z_k)p(x_{i}|z_k)}{\sum_{k'=1}^{K}p(z_{k'})p(x_{i}|z_{k'})} =\frac{
\pi_k\mathcal{N}(x_{i}|\mathbf{\mu}_k,\mathbf{\varSigma}_k)
}{
\sum_{k'=1}^{K}\pi_{k'}\mathcal{N}(x_{i}|\mathbf{\mu}_{k'},\mathbf{\varSigma}_{k'})
}
\end{split}
\end{displaymath}

观测样本$\mathbf{x}$的对数似然函数为:
\begin{displaymath}
\begin{split}
\log p(\mathbf{x}) &=\log \Big{\{} \prod_{i=1}^N{p(x_i)}\Big{\}}\\
&=\sum_{i=1}^{N} \log \Bigg{\{} \sum_{k=1}^{K}\{p(x_i,z_k)\}
\Bigg{\}}\\
&=\sum_{i=1}^{N} \log \Bigg{\{} \sum_{k=1}^{K}\{\mathcal{Q}(z_k)\frac{p(x_i,z_k)}{\mathcal{Q}(z_k)}\} \Bigg{\}}\\
&\geq \sum_{i=1}^{N} \sum_{k=1}^{K} \Bigg{\{}
\mathcal{Q}(z_k) \log \{ \frac{p(x_i,z_k)} {\mathcal{Q}(z_k)}  \}
\Bigg{\}}\\
\end{split}
\end{displaymath}

Jessen不等式成立的条件是对每个不同的$i$,$\frac{p(x_i,z_k)} {\mathcal{Q}(z_k)} $为常数
$C_i$,故$p(x_i,z_k) =C_i \times \mathcal{Q}(z_k)$:

\begin{displaymath}
\begin{split}
\sum_{k=1}^{K}p(x_i,z_k) =C_i \times \sum_{k=1}^{K}\mathcal{Q}(z_k)
&\Longrightarrow C_i = \sum_{k=1}^{K}p(x_i,z_k) = p(x_i) \\
\mathcal{Q}(z_k) = \frac{p(x_i,z_k)}{p(x_i)} &= p(z_k|x_i) = w_{k,i}
\end{split}
\end{displaymath}
故而得到E步的更新公式为:
\begin{displaymath}
\begin{split}
\mathcal{Q}(z_k) &= w_{k,i} = p(z_k|x_i)\\
&= \frac{
\pi_k\mathcal{N}(x_{i}|\mathbf{\mu}_k,\mathbf{\varSigma}_k)
}{
\sum_{k'=1}^{K}\pi_{k'}\mathcal{N}(x_{i}|\mathbf{\mu}_{k'},\mathbf{\varSigma}_{k'})
}
\end{split}
\end{displaymath}

我们现在最大化由Jessen不等式得到的下界:
\begin{displaymath}
\begin{split}
max \Bigg{\{} 
\sum_{i=1}^{N} \sum_{k=1}^{K} \Bigg{\{}
w_{k,i} \log \{ \frac{p(x_i,z_k)} {w_{k,i}}  \}
\Bigg{\}}
\end{split}
\end{displaymath}
subject to:
\begin{displaymath}
\sum_{k=1}^{K}{\pi_k} = 1
\end{displaymath}

引入拉格朗日算子得:
\begin{displaymath}
\begin{split}
&\mathop{\arg\max}_{\pi,\mu, \varSigma} \Bigg{\{} 
\sum_{i=1}^{N} \sum_{k=1}^{K} \Bigg{\{}
w_{k,i} \log \{ \frac{p(x_i,z_k)} {w_{k,i}}  \}
-\lambda \Big{\{} \sum_{k=1}^{K}{\pi_k}-1 \Big{\}}
\Bigg{\}}\\
&=\mathop{\arg\max}_{\pi,\mu, \varSigma} \Bigg{\{} 
\sum_{i=1}^{N} \sum_{k=1}^{K} \Bigg{\{}
w_{k,i} \log \{ \frac{p(x_i|z_k)p(z_k)} {w_{k,i}}  \}
-\lambda \Big{\{} \sum_{k=1}^{K}{\pi_k}-1 \Big{\}}
\Bigg{\}}\\
&=\mathop{\arg\max}_{\pi,\mu, \varSigma} \Bigg{\{} 
\sum_{i=1}^{N} \sum_{k=1}^{K} 
\Bigg{\{}
w_{k,i} \log \{ \frac{\pi_k \mathcal{N}(x_{i}|\mu_k,\varSigma_k)} {w_{k,i}}  \}
-\lambda \Big{\{} \sum_{k=1}^{K}{\pi_k}-1 \Big{\}}
\Bigg{\}}\\
&=\mathop{\arg\max}_{\pi,\mu, \varSigma} \Bigg{\{} 
\sum_{i=1}^{N} \sum_{k=1}^{K} 
\Bigg{\{}
w_{k,i} \{ 
\log{\pi_k} + \log{\mathcal{N}(x_{i}|\mu_k,\varSigma_k)}-\log{w_{k,i}}  
\}
-\lambda \Big{\{} \sum_{k=1}^{K}{\pi_k}-1 
\Big{\}}
\Bigg{\}}\\
&=\mathop{\arg\max}_{\pi,\mu, \varSigma} \Bigg{\{} 
\sum_{i=1}^{N} \sum_{k=1}^{K} 
\Bigg{\{}
w_{k,i} \{ 
\log{\pi_k} + \log{\mathcal{N}(x_{i}|\mu_k,\varSigma_k)} 
\}
-\lambda \sum_{k=1}^{K}{\pi_k} 
\Bigg{\}}\\
\end{split}
\end{displaymath}

对$\pi_k$求导为0得:
\begin{displaymath}
\begin{split}
&\frac{\sum_{i=1}^{N}{w_{k,i}}}{\pi_k}-\lambda = 0\\
\Longrightarrow & \sum_{k=1}^{K} \sum_{i=1}^{N} {w_{k,i}} =
\sum_{k=1}^{K}{\lambda \pi_k}\\
\Longrightarrow & \lambda = \sum_{k=1}^{K} \sum_{i=1}^{N} {w_{k,i}} =
N\\
\Longrightarrow & \pi_k = \frac{1}{N}\sum_{i=1}^{N}{w_{k,i}}
\end{split}
\end{displaymath}


令 $C=exp\{-\frac{1}{2}(x_i-\mu_k)\varSigma_k^{-1}(x_i-\mu_k)^T\}$,则:
\begin{displaymath}
\begin{split}
&\frac{dC}{d\mu_k} = C \varSigma_k^{-1}(x_i-\mu_k)^T\\
&\frac{d \mathcal{N}(x_{i}|\mathbf{\mu}_k,\mathbf{\varSigma}_k)}{d\mu_k} = \mathcal{N}(x_{i}|\mathbf{\mu}_k,\mathbf{\varSigma}_k) \varSigma_k^{-1}(x_i-\mu_k)^T\\
\end{split}
\end{displaymath}



对$\mu_k$求导为0得:
\begin{displaymath}
\begin{split}
&\frac{
\delta{ \Bigg{(}
\sum_{i=1}^{N} \sum_{k=1}^{K} 
\Bigg{\{}
w_{k,i} \{ 
\log{\pi_k} + \log{\mathcal{N}(x_{i}|\mu_k,\varSigma_k)} 
\}
-\lambda \sum_{k=1}^{K}{\pi_k} \Bigg{\}} \Bigg{)}
}}{\delta{\mu_k}} = 0\\
\Longrightarrow& \frac{
\delta{ \Bigg{(}
\sum_{i=1}^{N} \sum_{k=1}^{K} 
\Bigg{\{}
w_{k,i} \log{\mathcal{N}(x_{i}|\mu_k,\varSigma_k)} 
\Bigg{\}} \Bigg{)}
}}{\delta{\mu_k}} = 0\\
\Longrightarrow& \frac{
\delta{ \Bigg{(}
\sum_{i=1}^{N}
\Bigg{\{}
w_{k,i} \log{\mathcal{N}(x_{i}|\mu_k,\varSigma_k)} 
\Bigg{\}} \Bigg{)}
}}{\delta{\mu_k}} = 0\\
\Longrightarrow& 
\sum_{i=1}^{N}
\Bigg{\{}
w_{k,i} 
\frac{
      \frac{
           \delta(  \mathcal{N}(x_{i}|\mu_k,\varSigma_k)  )
           }{\delta{\mu_k}}
}
{\mathcal{N}(x_{i}|\mu_k,\varSigma_k)} 
\Bigg{\}}
 = 0\\
\end{split}
\end{displaymath}

由于 $b\frac{d
  \mathcal{N}(x_{i}|\mathbf{\mu}_k,\mathbf{\varSigma}_k)}{d\mu_k} =
\mathcal{N}(x_{i}|\mathbf{\mu}_k,\mathbf{\varSigma}_k)
\varSigma_k^{-1}(x_i-\mu_k)^T$:
\begin{displaymath}
\begin{split}
\Longrightarrow& 
\sum_{i=1}^{N}
\Bigg{\{}
w_{k,i} 
\frac{
 \mathcal{N}(x_{i}|\mathbf{\mu}_k,\mathbf{\varSigma}_k) \varSigma_k^{-1}(x_i-\mu_k)^T
}
{\mathcal{N}(x_{i}|\mu_k,\varSigma_k)} 
\Bigg{\}}
 = 0\\
\Longrightarrow& 
\sum_{i=1}^{N}
\Bigg{\{}
w_{k,i}  \varSigma_k^{-1}(x_i-\mu_k)^T
\Bigg{\}}
 = 0\\
\Longrightarrow& 
\sum_{i=1}^{N}
\Bigg{\{}
w_{k,i}(x_i-\mu_k)
\Bigg{\}}
 = 0\\
\Longrightarrow& 
\sum_{i=1}^{N}
\Bigg{\{}
w_{k,i}x_i
\Bigg{\}}
 = \sum_{i=1}^{N}
\Bigg{\{}
w_{k,i}\mu_k
\Bigg{\}}\\
\Longrightarrow& 
\mu_k = \frac{
\sum_{i=1}^{N}  w_{k,i}x_i 
}{ 
\sum_{i=1}^{N} w_{k,i}
}\\
\end{split}
\end{displaymath}


先列几个下边会用到的求导公式:
\begin{displaymath}
\begin{split}
&\frac{d|M|}{dM} = |M|(M^{-1})^T\\
&\frac{\delta(a^TMb)}{dM} = ab^T\\
\end{split}
\end{displaymath}

令 $C=exp\{-\frac{1}{2}(x_i-\mu_k)\varSigma_k^{-1}(x_i-\mu_k)^T\}$,则:
\begin{displaymath}
\begin{split}
\frac{dC}{d\varSigma_k^{-1}} &= C (-\frac{1}{2})(x_i-\mu_k)(x_i-\mu_k)^T\\
\frac{d
  \mathcal{N}(x_{i}|\mathbf{\mu}_k,\mathbf{\varSigma}_k)}{d\varSigma_k^{-1}}
&= \frac{C}{(2\pi)^{\frac{D}{2}}}
\frac{\delta{(|\varSigma_k^{-1}|^{\frac{1}{2}})}}{\delta{\varSigma_k^{-1}}}
+ \frac{|\varSigma_k^{-1}|^{\frac{1}{2}}}{(2\pi)^{\frac{D}{2}}}
\frac{\delta{C}}{\delta{\varSigma_k^{-1}}}
\\
&= \frac{C}{(2\pi)^{\frac{D}{2}}}
\frac{|\varSigma_k^{-1}|^{-\frac{1}{2}}|\varSigma_k^{-1}|\varSigma_k}{2}
+ \frac{|\varSigma_k^{-1}|^{\frac{1}{2}}}{(2\pi)^{\frac{D}{2}}}
 C (-\frac{1}{2})(x_i-\mu_k)(x_i-\mu_k)^T
\\
&= \frac{C|\varSigma_k^{-1}|^{\frac{1}{2}}\varSigma_k}{2(2\pi)^{\frac{D}{2}}}
- \frac{C|\varSigma_k^{-1}|^{\frac{1}{2}}(x_i-\mu_k)(x_i-\mu_k)^T}
{2(2\pi)^{\frac{D}{2}}}
\\
&= \frac{C|\varSigma_k^{-1}|^{\frac{1}{2}} 
(\varSigma_k - (x_i-\mu_k)(x_i-\mu_k)^T)
}{2(2\pi)^{\frac{D}{2}}}
\\
&=  \frac{1}{2}\mathcal{N}(x_{i}|\mu_k,\varSigma_k)
(\varSigma_k - (x_i-\mu_k)(x_i-\mu_k)^T)
\\
\end{split}
\end{displaymath}

对$\varSigma_k$求导为0得:
\begin{displaymath}
\begin{split}
&\frac{
\delta{ \Bigg{(}
\sum_{i=1}^{N} \sum_{k=1}^{K} 
\Bigg{\{}
w_{k,i} \{ 
\log{\pi_k} + \log{\mathcal{N}(x_{i}|\mu_k,\varSigma_k)} 
\}
-\lambda \sum_{k=1}^{K}{\pi_k} \Bigg{\}} \Bigg{)}
}}{\delta{\varSigma_k}} = 0\\
\Longrightarrow& \frac{
\delta{ \Bigg{(}
\sum_{i=1}^{N} \sum_{k=1}^{K} 
\Bigg{\{}
w_{k,i} \log{\mathcal{N}(x_{i}|\mu_k,\varSigma_k)} 
\Bigg{\}} \Bigg{)}
}}{\delta{\varSigma_k}} = 0
\end{split}
\end{displaymath}

\begin{displaymath}
\begin{split}
\Longrightarrow& \frac{
\delta{ \Bigg{(}
\sum_{i=1}^{N}
\Bigg{\{}
w_{k,i} \log{\mathcal{N}(x_{i}|\mu_k,\varSigma_k)} 
\Bigg{\}} \Bigg{)}
}}{\delta{\varSigma_k}} = 0 \\
\Longrightarrow &
\sum_{i=1}^{N}
\Bigg{\{}
w_{k,i} 
\frac{
      \frac{
           \delta(  \mathcal{N}(x_{i}|\mu_k,\varSigma_k)  )
           }{\delta{\varSigma_k}}
}
{\mathcal{N}(x_{i}|\mu_k,\varSigma_k)} 
\Bigg{\}}
 = 0\\
\end{split}
\end{displaymath}

由于 $b\frac{d
  \mathcal{N}(x_{i}|\mathbf{\mu}_k,\mathbf{\varSigma}_k)}{d\varSigma_k} =
 \frac{1}{2}\mathcal{N}(x_{i}|\mu_k,\varSigma_k)
(\varSigma_k - (x_i-\mu_k)(x_i-\mu_k)^T)$:
\begin{displaymath}
\begin{split}
\Longrightarrow& 
\sum_{i=1}^{N}
\Bigg{\{}
w_{k,i} 
\frac{
\frac{1}{2}\mathcal{N}(x_{i}|\mu_k,\varSigma_k)
(\varSigma_k - (x_i-\mu_k)(x_i-\mu_k)^T)
}
{\mathcal{N}(x_{i}|\mu_k,\varSigma_k)} 
\Bigg{\}}
 = 0\\
\Longrightarrow& 
\sum_{i=1}^{N}
\Bigg{\{}
w_{k,i}\varSigma_k
\Bigg{\}}
 = \sum_{i=1}^{N}
\Bigg{\{}
w_{k,i}(x_i-\mu_k)(x_i-\mu_k)^T
\Bigg{\}}
\\
\Longrightarrow& 
\varSigma_k = \frac{
\sum_{i=1}^{N}  w_{k,i} (x_i-\mu_k)(x_i-\mu_k)^T
}{ 
\sum_{i=1}^{N} w_{k,i}
}\\
\end{split}
\end{displaymath}

综上,GMM模型使用EM进行训练时的步骤为:

E步:选择一个合适的隐状态的分布$\Theta(z_k)$,使得样本$x$的最大似然概率
等于Jesen不等式得到的下界,即$\frac{p(x_i,z_k)} {\mathcal{Q}(z_k)}$的期望。根据Jesen不等式成立的
条件,给定样本$\mathbf{x}$和参数$\pi_k, \mu_k, \varSigma_k$的情况下,隐状态的分布为:
\begin{displaymath}
w_{k,i} = \frac{
\pi_k\mathcal{N}(x_{i}|\mathbf{\mu}_k,\mathbf{\varSigma}_k)
}{
\sum_{k'=1}^{K}\pi_{k'}\mathcal{N}(x_{i}|\mathbf{\mu}_{k'},\mathbf{\varSigma}_{k'})
}
\end{displaymath} 

M步:最大化Jesen不等式得到的下界,来求得参数$\pi_k, \mu_k, \varSigma_k$,即:
\begin{displaymath}
\begin{split}
\pi_k &= \frac{1}{N}\sum_{i=1}^{N}{w_{k,i}}\\
\mu_k &= \frac{
\sum_{i=1}^{N}  w_{k,i}x_i 
}{ 
\sum_{i=1}^{N} w_{k,i}
}\\
\varSigma_k &= \frac{
\sum_{i=1}^{N}  w_{k,i} (x_i-\mu_k)(x_i-\mu_k)^T
}{ 
\sum_{i=1}^{N} w_{k,i}
}\\
\end{split}
\end{displaymath} 

