% !Mode:: "TeX:UTF-8"

\section{EM}


参数为$\theta$,样本$x_i$的概率为$p(x_i;\theta)$,隐变量为$z$,则最大似
然为:
\begin{displaymath}
\begin{split}
\ell(\theta) &= \sum_{i=1}^{n}{log(p(x_i;\theta))}\\
&= \sum_{i=1}^{n}{log(\sum_{z}{p(x_i,z;\theta)})}
\end{split}
\end{displaymath} 

令$\Theta(z)$为隐变量$z$的分布,则:
\begin{displaymath}
\begin{split}
\ell(\theta) &= \sum_{i=1}^{n}{log(\sum_{z}{p(x_i,z;\theta)})}\\
&=
\sum_{i=1}^{n}{log(\sum_{z}{(\Theta(z)*\frac{p(x_i,z;\theta)}{\Theta(z)}}))}\\
&\geq \sum_{i=1}^{n}{\sum_{z}{\Theta(z)*log(\frac{p(x_i,z;\theta)}{\Theta(z)}})}
\end{split}
\end{displaymath} 
最后一步为Jesen不等式。有公式可以看到,最大似然$\ell(\theta)$的下界为
$\frac{p(x_i,z;\theta)}{\Theta(z)}$的期望。

我们可以选择一个合适的隐状态的分布$\Theta(z)$使得不等式的等号成立,根
据Jesen不等式成立的条件,对每个不同的$i$,$log$内部的$\frac{p(x_i,z;\theta)}{\Theta(z)}$为常数
$C_i$,即$\frac{p(x_i,z;\theta)}{\Theta(z)}$不随$z$的变化而变化,得:
$\frac{p(x_i,z;\theta)}{\Theta(z)} = C_i$
而又有 $\sum_{z}{\Theta(z)}=1$,从而:
\begin{displaymath}
\sum_{z}{p(x_i,z;\theta)} = \sum_{z}{\Theta(z)*C_i}
=C_i
=p(x_i;\theta)
\end{displaymath} 
故而:
\begin{displaymath}
\begin{split}
\Theta(z) &= \frac{p(x_i,z;\theta)}{p(x_i;\theta)}\\
&=p(z|x_i;\theta)
\end{split}
\end{displaymath} 
即,当隐状态$z$的分布$\Theta(z)$取最大似然估计的结果$p(z|x_i;\theta)$时,
不等式的等号成立:样本$x$的最大似然概率等于
$\frac{p(x_i,z;\theta)}{\Theta(z)}$的期望。

目前为止,给定了样本$x$和参数$\theta$,我们可以用最大似然来获得隐状态
分布$\Theta(z)$,而给定了样本$x$和隐状态的分布$\Theta(z)$,我们又可以
通过最大化样本的似然概率来获得参数$\theta$。故而,EM训练的过程为:

E步:选择一个合适的隐状态的分布$\Theta(z)$,使得样本$x$的最大似然概率
等于Jesen不等式得到的下界,即$\frac{p(x_i,z;\theta)}{\Theta(z)}$的期望。根据Jesen不等式成立的
条件,给定样本$x$和参数$\theta$的情况下,隐状态的分布为:
\begin{displaymath}
\Theta(z)=p(z|x_i;\theta)
\end{displaymath} 

M步:最大化Jesen不等式得到的下界,来求得参数$\theta$,即:
\begin{displaymath}
\begin{split}
\theta &= argmax{\sum_{i=1}^{n}{\sum_{z}{\Theta(z)*log(\frac{p(x_i,z;\theta)}{\Theta(z)}})}}
\end{split}
\end{displaymath} 

Jessen不等式理解:令$\frac{p(x_i,z;\theta)}{\Theta(z)} = x_z$,则有:
\begin{displaymath}
\begin{split}
log(\sum_{z}{(\Theta(z)*\frac{p(x_i,z;\theta)}{\Theta(z)}})) &=log(\sum_{z}{(\Theta(z)*x_z}))\\
&=log(\Theta_{0}x_0 + \Theta_{1}x_1 +\Theta_{2}x_2+...+\Theta_{n}x_n)\\
&\geq \Theta_0log(x_0) + \Theta_1log(x_1) +\Theta_2log(x_2)+...+\Theta_nlog(x_n)\\
&=\sum_{z}{\Theta(z)*log(x_z)}\\
&=\sum_{z}{(\Theta(z)*log(\frac{p(x_i,z;\theta)}{\Theta(z)})})\\
\end{split}
\end{displaymath} 

当$x_z=C$,且$sum_{z}{\Theta(z)} = 1$(概率分布的性质)时,
\begin{displaymath}
\begin{split}
log(\sum_{z}{(\Theta(z)*x_z})) = log(1*C) &=log(C)\\
\sum_{z}{\Theta(z)*log(x_z)} = 1*log(C)&=log(C)
\end{split}
\end{displaymath} 
