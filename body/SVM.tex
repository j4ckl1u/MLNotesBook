% !Mode:: "TeX:UTF-8"

\chapter{SVM}

\section{不带松弛因子的SVM}
样本点$x_i$到直线$y=wx+b$的距离为:
\begin{displaymath}
\frac{wx_i+b}{\| w \|}
\end{displaymath}  
 
样本点$x_i$到直线$y=wx+b$的几何距离为:
\begin{displaymath}
\frac{y_i*(wx_i+b)}{\| w \|}
\end{displaymath} 

不失一般性的假设样本集合中距离分类面$y=wx+b$的距离最小的样本为
$(x_i,y_i)$,则假设其算术距离为1,即:
\begin{displaymath}
y_i*(wx_i+b)=1
\end{displaymath} 

那么,SVM的优化目标变为:$argmax{\frac{1}{\|w\|}}$
给定: $min_i(y_i*(wx_i+b)) = 1$
即: $y_i*(wx_i+b) \geq 1$

又:$argmax{\frac{1}{\|w\|}} = argmin(\frac{1}{2}\|w\|) =
argmin(\frac{1}{2}w^2)$

有带约束问题的极值:
\begin{equation}
argmin(\frac{1}{2}w^2 - \sum_i{\lambda_i(y_i*(wx_i+b)-1)})
\end{equation} 

其为凸函数,对$w$求导为0得:
\begin{equation}
w = \sum_i{\lambda_i(y_i*x_i)}
\end{equation} 

对$b$求导为0得:
\begin{equation}
 \sum_i{\lambda_i*y_i}=0
\end{equation}

将求得的$w$代入原公式的:
\begin{equation}
\begin{split}
&\frac{1}{2}w^2 - \sum_i{\lambda_i(y_i*(wx_i+b)-1)}\\
&=\frac{1}{2}(\sum_i{\lambda_i(y_i*x_i)})(\sum_j{\lambda_j(y_j*x_j)}) -
\sum_i{\lambda_i(y_i*((\sum_i{\lambda_j(y_j*x_j)})x_i+b)-1)}\\
&=-\frac{1}{2}\sum_i\sum_j{\lambda_i\lambda_jy_iy_jx_ix_j}
-\sum_i{\lambda_i(y_ib)}+\sum_i{\lambda_i}\\
&\sum_i{\lambda_i*y_i}=0\\
&=\sum_i{\lambda_i}-\frac{1}{2}\sum_i\sum_j{\lambda_i\lambda_jy_iy_jx_ix_j}
\end{split}
\end{equation}

故原始最优化问题转换为:
\begin{equation}
argmin_\lambda{\sum_i{\lambda_i}-\frac{1}{2}\sum_i\sum_j{\lambda_i\lambda_jy_iy_jx_ix_j}}
\end{equation}
subject to: $ \sum_i{\lambda_i*y_i}=0$ and $\lambda_i \geq 0$。称为原
始问题的对偶形式。

该受限极值问题为二次规划问题,可以用 Sequential Minimal Optimization
(SMO)方法求解。

\section{带松弛因子的SVM}
为了处理线性不可分的情况,引入松弛变量$\xi \geq 0$,从而允许某些样本点到分类面的函数距离少于1,即
\begin{equation}
 y_i*(wx_i+b) \geq 1-\xi_i
\end{equation}
注意,每个样本点可选择不同的$\xi$,该参数会被模型优化。

为了使得松弛变量$\xi_i$越小越好,将优化目标变为:
\begin{displaymath}
argmin_w{\frac{1}{2}*w^2} + C\sum_i{\xi_i}
\end{displaymath} 

引入拉格朗日算子,将受限极值转换为非受限极值:
\begin{equation}
argmin{\frac{1}{2}w^2  + C\sum_i{\xi_i} -
\sum_i{\lambda_i(y_i*(wx_i+b)-1+\xi_i)} - \sum_i{\beta_i\xi_i}}
\end{equation} 

函数为凸,对$w$,$b$,$\xi$求导为0得:
\begin{equation}
w = \sum_i{\lambda_i(y_i*x_i)}
\end{equation} 
\begin{equation}
 \sum_i{\lambda_i*y_i}=0
\end{equation}

\begin{equation}
C-\lambda_i-\beta_i=0 \Longrightarrow
C-\lambda_i = \beta_i
\end{equation}

将求导所得代入原始公式得原始问题的对偶形式:
\begin{equation}
argmin_\lambda{\sum_i{\lambda_i}-\frac{1}{2}\sum_i\sum_j{\lambda_i\lambda_jy_iy_jx_ix_j}}
\end{equation}
subject to:
\begin{equation}
\begin{split} 
\sum_i{\lambda_i*y_i} &=0\\
\lambda_i &\geq 0 \\
\left\{
\begin{aligned}
 C-\lambda_i &=& \beta_i\\
\beta_i &\geq& 0
\end{aligned}
\right\}
\Longrightarrow
C-\lambda_i \geq 0 \Longrightarrow \lambda_i \leq C
\end{split}
\end{equation}

比较引入松弛变量$\xi_i$的最后结果同引入前的相比,差别仅仅在于对
$\lambda_i$的约束上,最后的模型并没有保留任何松弛变量。
