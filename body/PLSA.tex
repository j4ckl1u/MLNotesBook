% !Mode:: "TeX:UTF-8"

\section{PLSA}


样本为词和文档的一个pair $<w,d>$,隐变量为topic $z$。似然函数的计算为:
\begin{displaymath}
\begin{split}
\ell &= \sum_{w,d}{log(p(w,d))}\\
&= \sum_{w,d}{log(p(w|d)) + log(p(d))}\\
&= \sum_{w,d}{log(\sum_z{p(w,z|d)})+ log(p(d))}\\
&= \sum_{w,d}{log(\sum_z{p(w|z)p(z|d)})+ log(p(d))}
\end{split}
\end{displaymath} 

引入隐变量$\mathcal{Q}(z)$,并应用Jessen不等式得:
\begin{displaymath}
\begin{split}
\ell' &= \sum_{w,d}{log(\sum_z{p(w|z)p(z|d)})}\\
&=\sum_{w,d}{log(\sum_z{\frac{p(w|z)p(z|d)}{\mathcal{Q}(z)}*\mathcal{Q}(z)})}\\
&\geq \sum_{w,d}{\sum_z{\mathcal{Q}(z)*log(\frac{p(w|z)p(z|d)}{\mathcal{Q}(z)}}))}
\end{split}
\end{displaymath} 

Jessen不等式成立的条件为 $\frac{p(w|z)p(z|d)}{\mathcal{Q}(z)}$与$z$无
关:
\begin{displaymath}
\begin{split}
\frac{p(w|z)p(z|d)}{\mathcal{Q}(z)}&=C_{w,d}\\
\sum_z{p(w|z)p(z|d)}=p(w|d) &=\sum_z{C_{w,d}*\mathcal{Q}(z)}=C_{w,d}\\
p(w|d)=C_{w,d} \Longrightarrow
\mathcal{Q}(z)&=\frac{p(w|z)p(z|d)}{p(w|d)} = p(z|w,d)
\end{split}
\end{displaymath} 
从而得到E步的更新公式:
\begin{displaymath}
\begin{split}
p(z|w,d) = \frac{p(w|z)p(z|d)}{p(w|d)}
\end{split}
\end{displaymath} 
经过E步得到$p(z|w,d)$后,M步将最大化Jessen不等式得到的似然函数的下界。
由于$p(z|w,d)$已知,故公式可化为:
\begin{displaymath}
\begin{split}
\mathop{\arg\max}_{p(w|z),p(z|d)} &\big\{
  \sum_{w,d}{
            \sum_z{
                   p(z|w,d)*log(
                                \frac{p(w|z)p(z|d)}{p(z|w,d)}
                                )
                  }
           } 
\big\}\\
=\mathop{\arg\max}_{p(w|z),p(z|d)} &\big\{
  \sum_{w,d}{\sum_z{p(z|w,d)*[log(p(w|z))+log(p(z|d)) -log(p(z|w,d))]}} 
\big\}\\
=\mathop{\arg\max}_{p(w|z),p(z|d)} &\big\{
  \sum_{w,d}{
            \sum_z{p(z|w,d)*[log(p(w|z))+log(p(z|d))]
        -p(z|w,d)*log(p(z|w,d))}} 
\big\}\\
=\mathop{\arg\max}_{p(w|z),p(z|d)} &\big\{
  \sum_{w,d}{
            \sum_z{
                   p(z|w,d)*[log(p(w|z))+log(p(z|d))]
                  }
           } 
\big\}
\end{split}
\end{displaymath} 

object to:
\begin{displaymath}
\begin{split}
\sum_w{p(w|z)} &= 1\\
\sum_z{p(z|d)} &= 1
\end{split}
\end{displaymath} 

引入拉格朗日算子$\lambda_z$和$\lambda_d$,将约束极值问题转换为非约束极
值问题:
\begin{displaymath}
\begin{split}
\mathop{\arg\max}_{p(w|z),p(z|d)} &\big\{
  \sum_{w,d}{
            \sum_z{
                   p(z|w,d)*[log(p(w|z))+log(p(z|d))]
                  }
           }\\
    &-\sum_z{\lambda_z[\sum_w{p(w|z)}-1]}
             -\sum_d{\lambda_d[\sum_z{p(z|d)}-1]}
\big\}
\end{split}
\end{displaymath}

分别对$p(w|z)$和$p(z|d)$求导得到$\lambda_z$和$\lambda_d$:
\begin{displaymath}
\begin{aligned}
\sum_d{p(z|w,d)} &= \lambda_z*p(w|z)\\
\sum_w{p(z|w,d)} &= \lambda_d*p(z|d)\\
\end{aligned} \} \Longrightarrow
\{
\begin{aligned}
\lambda_z &= \sum_w{\sum_d{p(z|w,d)}}\\
\lambda_d &= \sum_z{\sum_w{p(z|w,d)}}
\end{aligned}
\end{displaymath} 

从而得到$p(w|z)$和$p(z|d)$在M步的更新公式:
\begin{displaymath}
\begin{split}
p(w|z) &= \frac{\sum_d{p(z|w,d)}}{\lambda_z}
=\frac{
          \sum_d{p(z|w,d)}
        }{
           \sum_w{
                 \sum_d{p(z|w,d)}
                }
        }\\
p(z|d) &= \frac{\sum_w{p(z|w,d)}}{\lambda_d}
= \frac{\sum_w{p(z|w,d)}}{\sum_z{\sum_w{p(z|w,d)}}}
\end{split}
\end{displaymath} 


